\documentclass[12pt]{beamer}
\usetheme{UTMath}

\usepackage{amssymb,amsmath, amsfonts, graphicx, wasysym, moresize, url, setspace, float, lscape, afterpage, indentfirst, listings, color}

\usepackage{xcolor,colortbl}

\graphicspath{{images/}}

\newcommand\blfootnote[1]{%
  \begingroup
  \renewcommand\thefootnote{}\footnote{#1}%
  \addtocounter{footnote}{-1}%
  \endgroup
}

\title{Predicting Economic Recessions}

\author[Oliver Titus]{By Oliver Titus \\
\and Sponsored by Dr. Staab}
\institute{FSU}
\date{December 19, 2019}
\begin{document}
\begin{frame}
\titlepage
\includegraphics[scale=0.1]{logo.jpg}
\end{frame}

\begin{frame}
\frametitle{Introduction}
\begin{itemize}
\item What is an economic recession?
\pause
\begin{itemize}
\item In general, they are a significant decline in economic activity according to NBER. \pause
\item A recession begins when the business cycle goes from peak to trough. \pause
\item They are influenced by:
\begin{itemize}
\item GDP
\item real income
\item employment level
\item industrial production
\item wholesale-retail sales
\end{itemize}
\end{itemize} \pause
\item Used interest rate data from the Federal Reserve Bank of St. Louis and the Board of Governors of the Federal Reserve System to to make predictions of future economic recessions in the U.S. and determine which variables are important.
\end{itemize}
\end{frame}

\begin{frame}
\frametitle{Literature Review}
\begin{itemize}
\item Some models used by other economic researchers:
\begin{itemize}
\item simple rules of thumb
\item GDP forecasting model
\item Stock and Watson model
\item Logistic models
\item Boosting Regression Trees (BRT)
\item the Probit model
\end{itemize} \vfill \pause
\item Many researchers find that the leading indicator for economic recession is the term spread.
\end{itemize}
\end{frame}

\begin{frame}
\frametitle{Data: Summary Statistics}
\begin{table}[H]
\centering
\resizebox{\columnwidth}{!}{%
\begin{tabular}{|l|l|l|l|l|}
\hline
\rowcolor{gray} \textbf{Variable}               & \textbf{Mean} & \textbf{Std. Dev.} & \textbf{Min} & \textbf{Max} \\ \hline
\rowcolor{orange} NBER Recession Indicator        & 0.129         & 0.335              & 0            & 1            \\ \hline
Three Month Treasury Yield (\%) & 4.502         & 3.226              & 0.01         & 17.237       \\ \hline
Ten Year Treasury Yield (\%)   & 5.839         & 2.852              & 1.5          & 15.32        \\ \hline
\rowcolor{cyan} Spread (\%)                     & 1.337         & 1.236              & -3.505       & 4.146        \\ \hline
\rowcolor{cyan} Federal Funds Rate (\%)             & 4.785         & 3.595              & 0.07         & 19.1         \\ \hline
\rowcolor{cyan} Consumer Price Index            & 119.33        & 75.864             & 26.71        & 256.36       \\ \hline
\rowcolor{cyan} Industrial Production Index     & 63.73         & 28.505             & 18.61        & 110.55       \\ \hline
\rowcolor{cyan} Election Year & 0.246 & 0.431 & 0 & 1 \\ \hline
\end{tabular}%
} \\
\begin{flushleft}
\footnotesize \textbf{n}=783
\end{flushleft}
\end{table}
\begin{itemize}
\item Note: data from April 1953 to September 2019 was used to make predictions.
\end{itemize}
\end{frame}
\begin{frame}
\frametitle{Data: A plot of the Term Spread}
\centering
\includegraphics[scale=0.5]{spreadty3m.png}
\end{frame}
\begin{frame}
\frametitle{Methods: The Probit Model}
\begin{itemize}
\item The model used to make our prediction is a Probit equation given by: \pause
\begin{center}
{\scriptsize
\begin{align*}
 P(USREC_{t+h}) = F(\alpha + \beta_1 \; SPREAD_t + \beta_2 \; FFR_t + \beta_3 \; CPI_t + \beta_4 INDPRO_t + \beta_5 E_t)
\end{align*}} \pause
where $F$ is the standard normal cumulative density function which can represented as:
\begin{equation*}
F(z) = \int_{-\infty}^{z} \frac{1}{\sqrt{2\pi}}e^{-x^2/2} dx.
\end{equation*} 
\end{center}
\end{itemize}
\end{frame}

\begin{frame}
\frametitle{Methods: Plots of the Standard Normal CDF and PDF}
\begin{center}
\includegraphics[scale=0.33]{cdf.png}
\includegraphics[scale=0.33]{normalpdf.png}
\end{center}
\end{frame}

\begin{frame}
\frametitle{Methods: Model Selection and Performance}
\begin{itemize}
\item The models of best fit were selected based on minimizing the Akaike Information Criterion (AIC): \pause
\begin{align*}
\text{AIC}_i = 2K_i - 2\log(L_i)
\end{align*}
where $L_i$ is the maximum likelihood for the best model $i$ and $K_i$ is a penalty based on the number of variables included in model $i$. \vfill \pause
\item Performance was measured by finding the area under the Receiver Operating Characteristic Curve known as the AUC. 
\end{itemize}
\end{frame}

\begin{frame}
\frametitle{Results: Stepwise Probit Coefficients}
\begin{table}[H]
\centering
\resizebox{\columnwidth}{!}{%
\begin{tabular}{|l|l|l|l|l|}
\hline
\multicolumn{1}{|c|}{\textbf{Forecast Horizon}} & \multicolumn{1}{c|}{\textbf{\begin{tabular}[c]{@{}c@{}}Three Months\\ (AIC = 536.35)\end{tabular}}} & \multicolumn{1}{c|}{\textbf{\begin{tabular}[c]{@{}c@{}}Six Months\\ (AIC = 492.91)\end{tabular}}} & \multicolumn{1}{c|}{\textbf{\begin{tabular}[c]{@{}c@{}}One Year\\ (AIC = 442.47)\end{tabular}}} & \multicolumn{1}{c|}{\textbf{\begin{tabular}[c]{@{}c@{}}Two Years\\ (AIC = 535.41)\end{tabular}}} \\ \hline
Intercept                                       & \begin{tabular}[c]{@{}l@{}}-1.259***\\ (0.292)\end{tabular}                                         & \begin{tabular}[c]{@{}l@{}}-1.269***\\ (0.163)\end{tabular}                                       & \begin{tabular}[c]{@{}l@{}}-0.856***\\ (0.184)\end{tabular}                                     & \begin{tabular}[c]{@{}l@{}}-1.646***\\ (0.258)\end{tabular}                                      \\ \hline
Ten-Year less Three Month Spread                & \begin{tabular}[c]{@{}l@{}}-0.174**\\ (0.059)\end{tabular}                                          & \begin{tabular}[c]{@{}l@{}}-0.326***\\ (0.062)\end{tabular}                                       & \begin{tabular}[c]{@{}l@{}}-0.632***\\ (0.081)\end{tabular}                                     & \begin{tabular}[c]{@{}l@{}}-0.195***\\ (0.055)\end{tabular}                                      \\ \hline
Federal Funds Rate                              & \begin{tabular}[c]{@{}l@{}}0.102***\\ (0.019)\end{tabular}                                          & \begin{tabular}[c]{@{}l@{}}0.082***\\ (0.020)\end{tabular}                                        & \begin{tabular}[c]{@{}l@{}}0.036\\ (0.023)\end{tabular}                                         &                                                                                                  \\ \hline
Consumer Price Index                            & \begin{tabular}[c]{@{}l@{}}0.006\\ (0.004)\end{tabular}                                             &                                                                                                   &                                                                                                 & \begin{tabular}[c]{@{}l@{}}-0.021***\\ (0.004)\end{tabular}                                      \\ \hline
Industrial Production Index                     & \begin{tabular}[c]{@{}l@{}}-0.016\\ (0.011)\end{tabular}                                            &                                                                                                   &                                                                                                 & \begin{tabular}[c]{@{}l@{}}0.050***\\ (0.011)\end{tabular}                                       \\ \hline
Election Year                                   &                                                                                                     &                                                                                                   & \begin{tabular}[c]{@{}l@{}}0.239\\ (0.157)\end{tabular}                                         &                                                                                                  \\ \hline
\end{tabular}%
} \\
\footnotesize '***' p $\approx$ 0, '**' p $\leq$ 0.001, '*' p $\leq$ 0.01. '.' p $\leq$ 0.05   \\
Standard errors are in parentheses. 
\end{table}
\end{frame}

\begin{frame}
\frametitle{Results: The Probability of Recession Three Months Ahead}
\centering
\includegraphics[scale=0.5]{stepthreemonth.png}
\end{frame}

\begin{frame}
\frametitle{Results: The Probability of Recession Six Months Ahead}
\centering
\includegraphics[scale=0.5]{stepsixmonth.png}
\end{frame}

\begin{frame}
\frametitle{Results: The Probability of Recession One Year Ahead}
\centering
\includegraphics[scale=0.5]{stepyear.png}
\end{frame}

\begin{frame}
\frametitle{Results: The Probability of Recession Two Years Ahead}
\centering
\includegraphics[scale=0.5]{steptwoyear.png}
\end{frame}

\begin{frame}
\frametitle{Results: ROC Curves for Stepwise Models}
\centering
\begin{table}
\resizebox{\columnwidth}{!}{%
\begin{tabular}{c c}
\tiny Three months: & \tiny Six months: \\
\includegraphics[scale=0.3]{modelC3moROC.png} &
\includegraphics[scale=0.3]{modelB6moROC.png} \\
\tiny AUC $=$ 72.15\% & \tiny AUC $=$ 79.88\%
\end{tabular}%
}
\end{table}
\end{frame}

\begin{frame}
\frametitle{Results: ROC Curves for Stepwise Models}
\centering
\begin{table}
\resizebox{\columnwidth}{!}{%
\begin{tabular}{c c}
\tiny One year: & \tiny Two years: \\
\includegraphics[scale=0.3]{yearsteproc.png} &
\includegraphics[scale=0.3]{twoyearsteproc.png} \\
\tiny AUC $=$ 87.12\% & \tiny AUC $=$ 74.88\%
\end{tabular}%
}
\end{table}
\end{frame}

\begin{frame}
\frametitle{Conclusion}
\begin{itemize}
\item The term spread was an important predictor for all forecast horizons.
\vfill
\item The federal funds rate was important for the three and six month forecast horizons.
\vfill
\item The CPI and Industrial Production Index were significant for the two-year horizon.
\vfill
\item The election year dummy was not significant.
\vfill
\item The model that performed best was the one-year forecast horizon model since it had the lowest AIC and highest AUC.
\end{itemize}
\end{frame}

\begin{frame}
\centering
\Huge Thank you! \\
\end{frame}
\end{document}